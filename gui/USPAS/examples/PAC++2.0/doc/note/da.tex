\documentstyle{article}
\begin{document}

\hspace{8cm}{\large RHIC/AP/51}
\vspace{3cm}

\begin{center}
{\LARGE
Application of a adifferential algebra approach to a RHIC helical dipole.
}

\vspace{0.6cm}

{\large Nikolay Malitsky }
\vspace{0.6cm}

{\large December 8, 1994}
\end{center}
\vspace{1.4cm}

\section{Introduction}
This paper describes an object-oriented method, that enables one
to obtain Taylor maps for arbitrary optical elements and include them 
in different accelerator algorithms. The approach is based on the differential algebraic (DA) technique, which in the accelerator physics
was suggested by Martin Berz and implemented by him in the program COSY INFINITY\cite{COSY}. 
In order to make an efficient use of DA operations,
COSY INFINITY includes a special DA precompiler, which transforms 
arithmetic operations containing DA variables into a sequence of calls to
FORTRAN subroutines. 
In ZLIB++\cite{ZLIB}, 
the object-oriented version of the numerical library for differential 
algebra, truncated power series and Taylor maps are considered as corresponding
C++ classes ${\bf ZSeries}$ and ${\bf ZMap}$ with overloaded assignment,
additive and multiplicative operators. These objects were implemented directly 
in the numerical integrator instead of DOUBLE variables and used to derive 
a Taylor map for the RHIC helical dipole.

\section{DA Integrator}

The particle motion in the magnetic field is described by the following 
set of equations, written in terms of the MAD coordinates:  \\
\begin{eqnarray} \label{eq:motion}
\frac{d}{ds}x & = &
\frac{(1+hx)}{\frac{p_s}{p_0}}\left (\frac{p_x}{p_0}\right) 
\nonumber \\ 
\frac{d}{ds} \left (\frac{p_x}{p_0} \right ) & = & \left 
[\frac{\vec{p}}{p_0}\times \left (\frac{e\vec{B}}{p_0  c}\right ) 
	  \right ]_x \frac{1+hx}{\frac{p_s}{p_0}} + h\frac{p_s}{p_0}
\nonumber \\ 
\frac{d}{ds}y & = &
	  \frac{(1+hx)}{\frac{p_s}{p_0}}\left (\frac{p_y}{p_0}\right ) \\ 
\frac{d}{ds} \left (\frac{p_y}{p_0} \right ) & = & \left 
	  [\frac{\vec{p}}{p_0}\times \left (\frac{e\vec{B}}{p_0 c}\right ) 
	  \right ]_y \frac{1+hx}{\frac{p_s}{p_0}} 
\nonumber \\
\frac{d}{ds} 
	  \sigma & = & \frac{1}{(v_0/c)} - \frac{l^{'}}{(v/c)}
\nonumber \\ 
\frac{d}{ds} \left (\frac{\Delta E}{p_0 c} \right ) & = & 0, 
\nonumber 
\end{eqnarray}
where 
\begin{eqnarray*}
\hspace{3.0 cm}
\frac{p_s}{p_0}  =  \sqrt{1+\frac{2}{\beta}(\frac{\Delta{E}}{p_0c}) 
+(\frac{\Delta{E}}{p_0c})^2 - (\frac{p_x}{p_0})^2 - (\frac{p_y}{p_0})^2 } \\ 
\end{eqnarray*}
A class template {\bf RKIntegrator} was developed
to perform the numerical integration through arbitrary magnetic field
simultaneously for real and DA variables. Instances of template class {\bf RKIntegrator} are 
initialized by some external function $elementField$, which will be used by 
the Runge-Kutta integrator to calculate the specific magnetic field:
\begin{eqnarray*}
& & {\bf RKIntegrator}{<ZSeries, ZMap>}       \\
& & \hspace{1. cm} elementDAIntegrator(elementField, elementParameters,step);
\end{eqnarray*}
These functions can be collected in one library and shared by different users. 
In accordance with the basic principles of the object-oriented platform for 
accelerator codes PAC++\cite{PAC}
the integration is considered as an action of one object on another and defined 
as a multiplicative operator:
\begin{eqnarray*}
& & {\bf Map} \, m; \, \,\, m = 1; \\
& & m = elementDAIntegrator*m;              
\end{eqnarray*}

\section{Input language}
The implementation of the DA integrator completes the object-oriented 
approach for the description of accelerator 
structure\cite{PAC}.
All lattice elements are considered as instances of a C++ class {\bf Element}
and divided in three categories: {\bf MAD}, {\bf COSY} and {\bf WILD}.
\begin{itemize}
\item
{\bf MAD} elements form the majority of all elements and can be defined as the 
superposition of standard MAD parameters: \\
\begin{tabular}{c} \\
 {\bf Element} $ hb$ = $length$*${\bf L}$ + $2*PI/N$*${\bf ANGLE}$;\\
\end{tabular} \\ \\
\noindent
where element $hb$ is an object with length equal to  $length$ $m$ and 
bend angle $2*PI/N$ $rad$.
\item
{\bf COSY} elements include  "nonstandard" parameters, but can be defined by 
a Taylor map (e.g. helical dipole): 
\begin{eqnarray*}
& & {\bf RKIntegrator}{<ZSeries, ZMap>}       \\
& & \hspace{0.5 cm} elementDAIntegrator(elementField, elementParameters,step); 
\\
& & {\bf Element} \, \, helix = helixDAIntegrator*map;
\end{eqnarray*}
The inclusion of the DA integrator in the object-oriented input 
language enables one to inherit the flexibility of COSY INFINITY.
On the other hand, we keep the clarity of lattice description, 
because the accelerator  or transfer line usually contains only a few 
"nonstandard" elements.
\item {\bf WILD} elements. This category contains elements, which
cannot be completely described by the Taylor map (such as internal
target, splitter magnet etc.) At the optical design level
they must be replaced by elements of 1st or 2nd categories (e.g. target
as drift): \\
$
\begin{array}{lll} \\
{\bf Element} & target     & = lengthOfTarget*{\bf L}; \\
{\bf Teapot}  & ring       & = ...*target*...;   \\
\end{array}  \\ \\
$
where the class {\bf Teapot} is the object-oriented version of the program 
TEAPOT\cite{TEAPOT}.
The particle-target interactions may be described by some external class
{\bf Target} and included in the general numerical simulation with the 
overloaded
multiplicative operator $Target::operator*(Particle\& \, particle)$:
\begin{eqnarray*}
& & {\bf Target} \hspace{0.5 cm} tooth(toothParameters);  \\
& & ring.track(1, numberOfTarget-1, particle);            \\
& & particle = tooth*particle;                            \\
& & ring.track(numberOfTarget+1, numberOfLastElement, particle);
\end{eqnarray*}
This approach enables one also to use the different object-oriented HEP
libraries (as GIZMO) or test the new complicated algorithms without  
changing of the object-oriented accelerator programs. 
\end{itemize}

\section{RHIC helical dipole}
The field $\vec{B}$ in the current-free region of a helical magnet can
be expressed in Cartesian coordinate system
as:
\begin{eqnarray} \label{eq:Br}
B_x & = & B_r cos({\phi}) - B_{\phi} sin({\phi}) \nonumber \\
B_y & = & B_r sin({\phi}) + B_{\phi} cos({\phi}) \\
B_z & = & B_z                                    \nonumber 
\end{eqnarray}
where\cite{PTITSIN}
\begin{eqnarray*} 
B_r & = & -k\sum_{m=1}^{\infty}m I_m^{'}(mkr)(a_m cos(m\theta) +
                         b_m sin(m\theta)) \nonumber \\
B_z & = &  k\sum_{m=1}^{\infty}m I_m(mkr) (b_m cos(m\theta) -
                         a_m sin(m\theta)) \\
B_{\phi} & = & -\frac{1}{kr}B_z  \nonumber
\end{eqnarray*}
and $\theta = \phi - (kz + \phi_0) $, $x = rcos(\phi )$, $y = rsin(\phi )$.
Unfortunately, these equations cannot be expressed directly in  DA 
variables, because the inversion and square root functions for a DA variable
$r=(r_0,r_1,r_2,...,r_N)$ is defined as truncated power 
series\cite{BERZ}: 
\begin{eqnarray*}
& &  \sqrt{(r_0,r_1,r_2,...,r_N)} \\
& & =\sqrt{r_0}\sqrt{1+(0,\frac{r_1}{r_0},\frac{r_2}{r_0},...,\frac{r_N}{r_0})} \\
& & =\sqrt{r_0}\sum_{i=0}^{n}(-1)^{i}
\frac{1\cdot 3\cdot ...(2i-3)}{2\cdot 4\cdot ...(2i)}
(0,\frac{r_1}{r_0},\frac{r_2}{r_0},...,\frac{r_N}{r_0})^{i}
\end{eqnarray*}
and depends on $r_0$.
The expressions $r^{2n}$, $r^ncos(n\phi)$, and $r^nsin(n\phi)$  can be presented
as simple functions of x and y variables. 
To extract them the equations (\ref{eq:Br}) were 
transformed and written in the following form:
\begin{eqnarray} \label{eq:Bx}
B_x & = & \~{B}_r x - \~{B}_{\phi} y       \nonumber \\ 
& & +\sum_{m=1} \frac{(m k/2)^{m}}{(m-1)!}r^{m-1}
\{ b_m sin(\phi - m\theta) - a_m cos(\phi - m\theta)\} \nonumber \\
B_y & = & \~{B}_r y + \~{B}_{\phi} x       \nonumber \\ 
& & -\sum_{m=1} \frac{(m k/2)^{m}}{(m-1)!}r^{m-1}
\{ b_m cos(\phi - m\theta) + a_m sin(\phi - m\theta)\} \\
B_z & = & - k\~{B}_{\phi} r^{2} \nonumber \\
& & +k\sum_{m=1} \frac{(m k/2)^{m}}{(m-1)!}r^{m}
\{ b_m cos(m\theta) - a_m sin(m\theta) \} \nonumber
\end{eqnarray}
where
\begin{eqnarray*} 
\~{B}_r & = &  -8 \sum_{m=1} (km/2)^{m+2}
(\~{I}_{m-1}(mkr) - \frac{m}{2}\~{I}_{m}(mkr))\\
& & \cdot r^{m} \{ a_m cos(m\theta) + b_m sin(m\theta)\}   \\
\~{B}_{\phi} & = & -\frac{8}{k} \sum_{m=1} (km/2)^{m+3} \~{I}_m(mkr) \\
& & \cdot r^{m} \{ b_m cos(m\theta) - a_m sin(m\theta)\}
\end{eqnarray*}
and
\begin{eqnarray*} 
\~{I}_m(z)   =  \sum_{i=1}^{\infty}\frac{(z/2)^{2(i-1)}}{4\,i!\,(m+i)!} \\
\end{eqnarray*}
The two sets of equations (\ref{eq:Br}) and (\ref{eq:Bx}) were implemented 
in C++ and tested together in DOUBLE variables. When a perfect
agreement between different functions was achieved, the last one was accepted 
as the template function $helixField$ and located in the file $Field.hh$. 
The nominal design for the RHIC helical 
snake\cite{SNAKE} consists of 4  helical dipoles of 2.4 m length, the $B_0$ 
field for the outer modules is 1.458 T and for the inner ones is 4 T. 
The Taylor map of one helical dipole was obtained by the short program 
presented in Figure 1.
As a first step,
only the influence of the main harmonic $b_1$ was considered.
Results obtained with this approach agree with the first and second order transfer matrices derived by the SNIG program\cite{SNIG} via numerical 
integration of particle trajectories (see Figure 2). The input language
described in Section 3 provides one with several methods to include 
the helical snake in the Teapot tracking procedure. The easiest one is 
presented in Figure 3. 


\section{Acknowledgment}
I would like to thank R.Talman, S.Peggs, F.Pilat, and V.Ptitsin
for many useful discussions.

\begin{thebibliography} 
\noindent 
\bibitem{COSY} 
    M. Berz, "User's Guide and Reference Manual to COSY INFINITY v.6". 
 \bibitem{ZLIB} 
    N. Malitsky, A. Reshetov, and Y. Yan, SSCL-659, 1994.
\bibitem{PAC}
    N. Malitsky, A. Reshetov, G. Bourianoff, SSCL-675, 1994. 
\bibitem{TEAPOT} 
    L. Schachinger and R. Talman, Particle Accelerator,22,35(1987).
\bibitem{PTITSIN}
    V. Ptitsin, Note RHIC/AP/41(oct. 10,1994)
\bibitem{BERZ} 
    M. Berz, Particle Accelerators,1989,Vol.24,pp.109-124
\bibitem{SNAKE}
    A. Luccio, Presented at the Spin Accelerator Meeting, BNL, October 6,1994.
\bibitem{SNIG} 
    A. Luccio, Private communication.

\end{thebibliography} 


\end{document}
